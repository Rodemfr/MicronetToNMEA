\documentclass{report}

\usepackage[utf8]{inputenc}
\usepackage[T1]{fontenc}
\usepackage{hyperref}
\usepackage{geometry}
\usepackage{graphicx}
\usepackage[T1]{fontenc}
\usepackage{ltablex}
\geometry{hmargin=2.5cm,vmargin=1.5cm}

\begin{document}

\title{MicronetToNMEA User Manual}

\chapter{Introduction}

\section{Overview}

MicronetToNMEA is an ESP32-based project that provides a low-cost NMEA 0183 converter for Raymarine's TackTick wireless instruments. Initially developed to understand the Micronet wireless protocol for logging wind and speed data, the project has since expanded to offer a wide range of functionalities.

Key capabilities include:
\begin{itemize}
\item Bidirectional NMEA 0183 streaming to a PC or tablet, conveying data such as depth, water speed, wind, magnetic heading, and GNSS information.
\item Transmitting heading data from an attached LSM303 compass to Micronet displays.
\item Sending GNSS data (position, time, date, COG, SOG) to both Micronet displays and an external NMEA device.
\item Relaying navigation data (BTW, DTW, XTE, ETA) from navigation software like OpenCPN, qtVlm, or AVnav to Micronet displays.
\end{itemize}

\section{Disclaimer}

\textbf{Important:} MicronetToNMEA is a hobbyist project. The hardware is not waterproofed or marine-grade and is not designed for high reliability in a harsh marine environment. It should \textbf{not} be used as a primary navigation tool.

The Micronet wireless protocol has been reverse-engineered. While much of it is understood, some aspects may be incomplete or misinterpreted. For certified, state-of-the-art navigation equipment, please consult an authorized Raymarine/TackTick reseller.

\section{Contributors}

\begin{itemize}
\item Ronan Demoment: Main author
\item Dietmar Warning: LSM303 drivers \& bugfixes
\item \href{https://github.com/j-lang}{j\_lang}: UBLOX M8N initialization code
\item Contributors to the YBW forum's Micronet thread: \href{https://forums.ybw.com/index.php?threads/raymarines-Micronet.539500/}{Micronet Thread}
\end{itemize}

\chapter{Hardware and Software Requirements}

\section{Required Hardware}

The minimum hardware required to build the MicronetToNMEA converter is an ESP32 development board and a CC1101 radio module.

\subsection{ESP32 Board}

This project requires an ESP32 variant that supports Bluetooth Classic SPP, specifically one based on the D0WD chip. \textbf{Newer ESP32 variants such as the S3, C2, C3, and H2 do not support Bluetooth Classic and are not compatible.}

The recommended and most straightforward option is an ESP32 WROOM DevKitC board, which is widely available and inexpensive. An example is the \href{https://www.upesy.fr/products/upesy-esp32-wroom-devkit-board}{uPesy Wroom DevKit}.

\begin{figure}[h]
\centering
\includegraphics[width=150mm]{ESP32_Pinout.png}
\caption{WROOM DevKitC Pinout}
\label{figure:esp32pinout}
\end{figure}

The project was developed on an older ESP32-WROOM-32 board (DevKitC v1) but should be compatible with newer ESP32-WROOM-32E boards (DevKitC v4).

\subsection{CC1101 Board}

The CC1101 is the radio transceiver that enables RF communication with Micronet/TackTick devices and is mandatory. While low-cost CC1101 breakout boards are common, their design and component quality can be inconsistent, potentially resulting in a shorter range compared to original TackTick equipment.

\textbf{Crucially, you must order a board designed for the 868/915MHz frequency range.} Although the CC1101 chip is multi-band, the board's antenna and filters are frequency-specific. Using a board for a different frequency (e.g., 434MHz) will severely limit the operational range. An example of a suitable board is this \href{https://www.amazon.fr/laqiya-cc1101-868-MHz-Transmission-Antenne-Transceiver/dp/B075PFQ57G}{868MHz CC1101 module}.

These boards often lack documentation. If no pinout is provided, refer to the official \href{https://www.ti.com/lit/ds/symlink/cc1101.pdf}{CC1101 datasheet}.

\section{Optional Hardware}

The converter's capabilities can be extended with the following optional hardware.

\subsection{NMEA 0183 GNSS Receiver}

You can connect any GNSS/GPS unit that outputs NMEA 0183 data over a serial (RS-232) connection. The UBLOX NEO-M8N is a good, low-cost example. When selecting a GNSS module, ensure its I/O voltage is 3.3V. \textbf{Connecting a 5V device will damage the ESP32's inputs, which are not 5V tolerant.} To avoid counterfeit hardware, purchase from reputable suppliers.

\subsection{NMEA 0183 AIS Unit}

An existing AIS unit with an NMEA 0183 output can serve as a GNSS source. MicronetToNMEA can also forward AIS data to a connected PC or tablet. Note that AIS NMEA 0183 outputs are typically differential and not 3.3V. An RS-422/485 to 3.3V RS232 transceiver may be required to protect the ESP32.

\subsection{LSM303 Compass Module}

An LSM303DLH, LSM303DLHC, or LSM303AGR compass module can be connected to the ESP32's I2C bus to provide magnetic heading data. The software automatically detects the presence and type of the connected LSM303 sensor.

\section{Required Software}

The project is developed using \href{https://code.visualstudio.com/}{Visual Studio Code} with the PlatformIO extension. All necessary configuration files are included in the repository for a straightforward setup.

\chapter{Compilation and Configuration}

\section{Compiling the Firmware}

Follow these steps to compile the MicronetToNMEA firmware:
\begin{itemize}
\item Install Visual Studio Code from your distribtion software manager or from the official website (\url{https://code.visualstudio.com}).
\item Launch Visual Studio Code and install the PlatformIO extension with the extension manager. On some Linux distribution, you may need to install the \emph{python3\_distutils} package first.
\item Open the MicronetToNMEA project folder using the \emph{"File -> Open Folder"} menu. This is the folder containing the \emph{platformio.ini} file.
\item Upon opening the project for the first time, PlatformIO will automatically download and install the required ESP32 toolchain. This may take several minutes.
\item Once the setup is complete, compile the project by pressing \emph{SHIFT+CTRL+B}.
\item Upload the compiled firmware to the ESP32 board by pressing \emph{SHIFT+CTRL+U}.
\end{itemize}

\section{Compile-Time Configuration}
\label{compile-time-configuration}

The file \emph{src/Config/BoardConfig.h} contains hardware layout definitions, such as pin assignments for SPI, I2C, and GPIO. You can modify this file to match your specific hardware setup. It is your responsibility to ensure the pin configuration is valid (e.g., assigning SPI functions to SPI-capable pins), as the software does not perform any validation.

Table \ref{table:configswitches} lists the available configuration options.

\begin{table}[h]
\begin{tabularx}{\linewidth}{@{}lX@{}}
\hline
\textbf{Switch} & \textbf{Description}\\
\hline
\texttt{FREQUENCY\_SYSTEM} & Sets the Micronet network frequency (0 for 868MHz, 1 for 915MHz). \\
\hline
\texttt{CS0\_PIN} & Defines the SPI Chip Select (CS) pin for the CC1101. \\
\hline
\texttt{MOSI\_PIN} & Defines the SPI MOSI pin for the CC1101. \\
\hline
\texttt{MISO\_PIN} & Defines the SPI MISO pin for the CC1101. \\
\hline
\texttt{SCK\_PIN} & Defines the SPI SCK (Clock) pin for the CC1101. \\
\hline
\texttt{GDO0\_PIN} & Defines the GDO0 pin for the CC1101. \\
\hline
\texttt{COMPASS\_I2C} & Sets the I2C bus for the LSM303 compass (e.g., \texttt{Wire0}, \texttt{Wire1}). \\
\hline
\texttt{BLUETOOTH\_DEVICE\_NAME} & Sets the Bluetooth device name. \\
\hline
\texttt{CONSOLE} & Defines the serial port for the configuration console (e.g., \texttt{USB\_NMEA}, \texttt{WIRED\_NMEA}). \\
\hline
\texttt{CONSOLE\_BAUDRATE} & Defines the baud rate for the console. \\
\hline
\texttt{NMEA0183\_IN\_IS\_UBLOXM8N} & Enables auto-configuration for a UBLOX M8N/M6N GPS (1 to enable, 0 to disable). \\
\hline
\texttt{NMEA0183\_IN} & Defines the serial port for NMEA input (e.g., \texttt{Serial}, \texttt{Serial1}). \\
\hline
\texttt{NMEA0183\_IN\_BAUDRATE} & Defines the baud rate for the NMEA input UART. \\
\hline
\texttt{PLOTTER} & Defines the serial port for the plotter/navigation computer (e.g., \texttt{gBTSerial}, \texttt{Serial}). \\
\hline
\end{tabularx}
\caption{Configuration switches in BoardConfig.h}
\label{table:configswitches}
\end{table}

\chapter{Hardware Installation}

Connect the ESP32 board to the peripheral modules according to the pin definitions in your \emph{BoardConfig.h} file. Double-check all connections before powering on the system, as incorrect wiring, especially for the power supply, can permanently damage the hardware.

\section{Power Supply}

There are two options for powering the system:
\begin{itemize}
\item Power via the ESP32's USB port.
\item Power from an external DC source.
\end{itemize}

\subsection{Powering via USB}

The simplest method is to power the system by connecting a USB cable from a PC to the ESP32 board. The board's onboard voltage regulator will provide the 3.3V needed by the other modules. Note that a standard USB 2.0 port is limited to 500mA. Ensure your system's total current draw does not exceed this limit.

Table \ref{table:boardconsumption} provides typical current consumption values.
\begin{table}[h]
\begin{tabular}{|l|c|c|l|}
\hline
Board & Supply & Max Current & Comment \\
\hline
ESP32 WROOM DevKitC & 5V & <100mA avg, 300mA peak & \\
CC1101 & 3.3V & 40mA & RF transmission at 868MHz \\
NEO-M8N GNSS & 5V & 45mA & I/O pins are 3.3V \\
LSM303DLH(C) & 3.3V & ~10mA & Assumed value \\
\hline
\end{tabular}
\caption{Typical current consumption of modules}
\label{table:boardconsumption}
\end{table}

USB power is convenient when using the USB serial output for NMEA data, as the connected PC or tablet powers the device.

\subsection{Powering with an External DC Source}

On a boat, power is typically supplied from a 12V battery system, which can have voltage fluctuations (11V-15V). To use this source, you must add a DC-DC converter to produce a stable 5V supply. This 5V output can be connected to the 5V input pin on the ESP32 board. The ESP32's onboard regulator will then generate the necessary 3.3V.

\begin{figure}[h]
\centering
\includegraphics[width=100mm]{MicronetToNMEA_DC_Power.png}
\caption{Powering the ESP32 with a DC-DC converter}
\label{figure:dcpower}
\end{figure}

\section{Connecting the CC1101 Module}

The CC1101 module operates at 3.3V. Connect its VCC and GND pins to the 3.3V and GND pins on the ESP32. Connect the MOSI, MISO, SCK, CS, and GDO0 pins as defined in \emph{BoardConfig.h}. The GDO2 pin is not used and can be left disconnected. Figure \ref{figure:cc1101} shows the default wiring scheme.

\begin{figure}[h]
\centering
\includegraphics[width=50mm]{MicronetToNMEA_CC1101.png}
\caption{Connecting the ESP32 and CC1101}
\label{figure:cc1101}
\end{figure}

\section{Connecting the LSM303 Module}

The LSM303 compass also operates at 3.3V. Connect its VCC and GND pins to the ESP32. Connect the SDA and SCL pins according to your \emph{BoardConfig.h} definitions. The DRDY, I1, and I2 pins are not used. Figure \ref{figure:lsm303} shows the default wiring.

\begin{figure}[h]
\centering
\includegraphics[width=50mm]{MicronetToNMEA_LSM303.png}
\caption{Connecting the ESP32 to the LSM303}
\label{figure:lsm303}
\end{figure}

\section{Connecting a GNSS Receiver}

Some GNSS/GPS boards require a 5V supply. If so, connect the board's VCC pin directly to the 5V output of your DC-DC converter. If your GNSS board is 3.3V powered, connect it to a 3.3V pin on the ESP32.

Connect the GNSS TX/RX pins to the corresponding RX/TX pins of the UART specified in \emph{BoardConfig.h}. The recommended UBLOX M8N module has an onboard regulator and can be powered with 5V, but its I/O are 3.3V tolerant. This may not be true for all GNSS modules, so always check that TX \& RX are 3.3V.

Figure \ref{figure:gnss} shows the default wiring.

\begin{figure}[h]
\centering
\includegraphics[width=50mm]{MicronetToNMEA_GNSS.png}
\caption{Connecting the ESP32 and GNSS}
\label{figure:gnss}
\end{figure}

MicronetToNMEA can interface with most GNSS receivers that output standard NMEA sentences at the baud rate specified in \emph{BoardConfig.h}. If you are using a UBLOX Neo M8N or M6N, you can enable the NMEA0183\_IN\_IS\_UBLOXM8N option in \emph{BoardConfig.h} to have the software configure it automatically.

The GNSS should be configured to output the following sentences:
\begin{itemize}
\item GGA: Position data
\item RMC: Time, date, and basic navigation data
\item VTG: Course over ground (COG) and speed over ground (SOG)
\end{itemize}

\section{Installation Recommendations}

\subsection{RF Performance}
Micronet devices operate at low power, with a typical range of 15-20 meters. To maximize performance, avoid placing metal objects or panels between MicronetToNMEA and other network devices. Carbon fiber, used in some sails and boat hulls, can also block RF signals and significantly reduce range. Fiberglass (GRP) hulls, however, attenuate the signal only slightly.

The CC1101's antenna quality is critical. The small antennas supplied with low-cost modules often prioritize size over performance. A simple quarter-wavelength wire antenna (86.5mm for 869MHz or 82mm for 915MHz) can provide excellent range. The \emph{"Test RF quality"} menu can be used to evaluate antenna performance and find the optimal orientation.

\subsection{Magnetic Compass Placement}
\label{compass-recommendations}
If using an LSM303 compass, install the MicronetToNMEA assembly far from electrical devices, especially power cables, which can create magnetic interference. For example, a 24" monitor can cause a 20° deviation at 50cm.

Also, avoid large metal objects such as the keel, batteries, and engine. On a smaller scale, keep the LSM303 board as far from the DC-DC converter as practical.

\subsection{Magnetic Compass Calibration}
For accurate heading measurements, the compass must be calibrated. The ideal calibration involves rotating the boat on all three axes, which is not realistic. A good alternative is to perform the calibration of MicronetToNMEA outside of the boat, on land, away from any metal or electronic devices.

Once calibrated, install the unit in its final position on the boat, following the placement recommendations above. This should yield good results.

\chapter{Usage}

\section{Initial Configuration}

\subsection{Connecting to the Console}

To configure MicronetToNMEA, you must connect to its serial console. By default, the console is accessible over the ESP32's USB serial port. Use a terminal application (such as Tera Term) to connect. When using the USB connection, the baud rate setting on the PC side does not matter.

On first power-up, the configuration menu should appear automatically. If the device instead begins outputting NMEA sentences, it means a configuration is already saved in EEPROM. Press the \emph{<ESC>} key to return to the main menu.

The menu will appear as follows:
\begin{verbatim}
*** MicronetToNMEA ***

0 - Print this menu
1 - General info on MicronetToNMEA
2 - Attach converter to closest network
3 - Start NMEA conversion
4 - Scan surrounding Micronet traffic
5 - Calibrate RF XTAL
6 - Calibrate compass
7 - Test RF quality
8 - Configuration
\end{verbatim}

\subsection{Attaching to a Micronet Network}

To begin, you must connect the MicronetToNMEA to your boat's Micronet network. This allows it to ignore signals from other nearby networks, a common situation in a marina.
Power on your Micronet system and place the master display (the one used to power on the network) within 2-3 meters of the MicronetToNMEA device. From the console menu, select the option \emph{"Attach converter to closest network"}. The device will scan for five seconds and list all detected networks, ordered by signal strength.

\begin{verbatim}
Network - 83038F54 (very strong)
Network - 810278A6 (low)
\end{verbatim}

Your network will likely be at the top of the list with a "very strong" signal. 
MicronetToNMEA will propose you to attach to this network. Select 'y' (yes) and the network ID will be saved to EEPROM. On subsequent startups, the device will automatically enter NMEA conversion mode.

\subsection{Calibrating the CC1101 RF Crystal}

The CC1101 module uses a crystal oscillator to generate its radio frequency. The precision of this crystal varies depending on the quality of the breakout board you use, so it must be calibrated to ensure it operates on the correct frequency for optimal range. While TackTick devices are factory-calibrated, you must perform this calibration yourself for your DIY unit.

Select menu option \emph{"Calibrate RF XTAL}. The console will display instructions: power on your Micronet network and place the master device within one meter of the MicronetToNMEA unit. Press any key to begin the two-minute calibration process. The console will display a series of dots and stars:

\begin{verbatim}
...................*.**********************.**.....................................
\end{verbatim}

Each character represents a tested frequency. A dot means no signal was received, while a star indicates a message was successfully received. The software identifies the center of the starred region as the optimal frequency.

At the end of the process, you will be prompted to save the new calibration value. Answer "yes" to store it in EEPROM. This calibration only needs to be done once.

\subsection{Calibrating the Navigation Compass}

An uncalibrated compass will yield highly inaccurate heading values. Calibration compensates for magnetic deviations caused by the sensor itself and nearby electronics (such as the ESP32, GNSS module, PCB tracks, etc.). The process determines the minimum and maximum magnetic field strength on each of the sensor’s three axes. This calibration step compensates MicronetToNMEA for its own magnetic distortions, but does not compensate for magnetic influences from your boat. The boat-specific environment will be handled later using Tacktick’s built-in calibration procedure.

Before starting, prepare your environment as described in Section \ref{compass-recommendations}. MicronetToNMEA must be calibrated outside the boat, as far as possible from any metallic or electrical objects—including your computer and its display. As a rule of thumb, maintaining at least two meters of separation is recommended.

Then select the menu option \emph{“Calibrate compass”}. The console will continuously display calibration data:

\begin{verbatim}
(0.2 -0.35 -0.17)
[0.20 0.98] [0.12 1.0] [-0.25 0.99]
  ^     ^     ^    ^
  |     |     |    |
  |     |     |    ------ Y Amplitude
  |     |     ----------- Y Offset
  |     |
  |     ----- X Amplitude
  ----------- X offset
\end{verbatim}

The first line shows the current magnetic field readings for the X, Y, and Z axes.
The second line displays two parameters for each axis: the center value ((max + min) / 2) and the amplitude (max - min).

To calibrate, slowly rotate the device to capture the minimum and maximum readings for each axis. There are six values to determine: Xmax, Xmin, Ymax, Ymin, Zmax, Zmin. The software automatically updates these min/max values as you move the device.

Earth’s magnetic field vector points generally toward the north, but it also tilts downward in the northern hemisphere and upward in the southern hemisphere. If you obtain the maximum value for one axis—for example, the X axis—it indicates that this axis is aligned with the Earth’s magnetic field. Note this orientation, then align the other axes similarly to obtain their respective maximum values (or minimum values if the orientation is reversed).

A successful calibration will produce similar amplitude values across all three axes. The center values, however, may differ significantly.

When you are finished, press \emph{<ESC>} and save the calibration data to EEPROM.

\subsection{Testing RF Performance}

Menu \emph{"Test RF quality"} helps you evaluate the connection quality to each device on the Micronet network, which is useful for finding the best installation location for the converter and the best antenna orientation. The menu displays a list of all network devices and their signal strength indicators:

\begin{verbatim}
81071e60 LNK=7.6 NET=9 Dual Display [M]
01071e77 LNK=7.2 NET=9 Hull
81037082 LNK=6.2 NET=7 Dual Display
83037737 LNK=6.2 NET=6 Wind Display
830ab252 LNK=6.2 NET=9 Wind Display
\end{verbatim}

\begin{itemize}
\item LNK: The signal strength of the device as received by MicronetToNMEA.
\item NET: The signal strength of the master display as received by that device.
\end{itemize}

A value of 9 is excellent, while a value below 3 is very low. Use the \emph{LNK} value to find the optimal position for the MicronetToNMEA unit. Use the \emph{NET} value to ensure all devices are properly receiving the master's signal. The list refreshes every second. The wind transducer is often the most problematic device due to signal blockage from the mast. If \emph{LNK} value is too low, you have to move MicronetToNMEA. If \emph{NET} value is too low, you have to move your master device.

\section{NMEA Conversion Mode}

Menu \emph{"Start NMEA conversion"} activates the core functionality of the device. In this mode, MicronetToNMEA begins bidirectionally translating data between the Micronet network and the NMEA data links.

By default, NMEA output is routed to the USB serial port, and you will see NMEA sentences appearing in the console. The device also listens for incoming NMEA sentences on the same port and transmits the relevant data to the Micronet network. This mode must be active for GNSS or other NMEA data to appear on your Micronet displays, and for display settings (like depth offset) to be saved.

If the device has been attached to a network, it will automatically enter this mode on power-up. You can exit to the main menu by pressing \emph{<ESC>}. Note that pressing \emph{<ESC>} will stop NMEA conversion mode.

In conversion mode, the device performs the following actions:
\begin{itemize}
\item Collect and decode GNSS related NMEA sentences from NMEA0183\_IN link (as per \emph{BoardConfig.h} definition)
\item Collect and decode navigation related NMEA sentences from PLOTTER link
\item Forward GNSS related NMEA sentences from NMEA0183\_IN link to PLOTTER link
\item Forward AIS related NMEA sentences from NMEA0183\_IN link to PLOTTER link if an AIS is used to get GNSS data
\item Collect and decode data from Micronet devices
\item Send all data collected from NMEA links to Micronet network
\item Send all data collected from Micronet devices to PLOTTER link
\item Compute heading from LSM303
\item Send heading data to Micronet network and PLOTTER link
\end{itemize}

\subsection{NMEA Sentences and Data Flow}\label{supportednmeasentences}

MicronetToNMEA can be configured to process data from several sources, or "links":

TODO : Ajouer un shéma global des entrées sortie de MNT

\begin{itemize}
\item \textbf{Micronet}: Bidirectional data link from/to the wireless Micronet network via the CC1101.
\item \textbf{NMEA0183\_IN}: The serial port connected to a GNSS or AIS unit. Data from this link is decoded and sent to the Micronet network and is also forwarded to the plotter link.
\item \textbf{Plotter}: The bidirectional serial port connected to an external device (e.g., a PC or chartplotter). Data received from this link is decoded and sent to the Micronet network.
\item \textbf{Compass}: The I2C bus connected to the optional LSM303 compass. Heading data from this link is sent to both the Micronet network and the plotter link.
\end{itemize}

Table \ref{table:nmeasentences} summarizes the supported NMEA sentences and their data flow.

\begin{table}[h]
\small
\begin{tabular}{|l|c|c|p{5cm}|}
\hline
\textbf{Sentence} & \textbf{Action}  & \textbf{Micronet Data} & \textbf{Possible Sources} \\
\hline
RMB & Decoded & XTE, DTW, BTW, VMGWP & Plotter \\
\hline
RMC & Decoded/Forwarded & TIME, DATE & NMEA0183\_IN, Plotter \\
\hline
GGA & Decoded/Forwarded & LAT, LON & NMEA0183\_IN, Plotter \\
\hline
GLL & Decoded/Forwarded & LAT, LON & NMEA0183\_IN, Plotter \\
\hline
VTG & Decoded/Forwarded & COG, SOG & NMEA0183\_IN, Plotter \\
\hline
MWV & Decoded/Encoded & AWA, AWS, TWA, TWS & Micronet, NMEA0183\_IN, Plotter \\
\hline
DPT & Decoded/Encoded & DPT & Micronet, NMEA0183\_IN, Plotter \\
\hline
MTW & Decoded/Encoded & STP & Micronet, NMEA0183\_IN, Plotter \\
\hline
VLW & Decoded/Encoded & LOG, TRIP & Micronet, NMEA0183\_IN, Plotter \\
\hline
VHW & Decoded/Encoded & SPD & Micronet, NMEA0183\_IN, Plotter \\
\hline
HDG & Decoded/Encoded & HDG & Micronet, NMEA0183\_IN, Plotter \\
\hline
XDR & Decoded & VCC & Micronet \\
\hline
VDM & Forwarded & None & NMEA0183\_IN \\
\hline
VDO & Forwarded & None & NMEA0183\_IN \\
\hline
\end{tabular}
\caption{Supported NMEA Sentences}
\label{table:nmeasentences}
\end{table}

\end{document}