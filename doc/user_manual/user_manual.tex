\documentclass{report}

\usepackage[T1]{fontenc}
\usepackage[american]{babel}
\usepackage{hyperref}
\usepackage{geometry}
\usepackage{graphicx}
\geometry{hmargin=2.5cm,vmargin=1.5cm}

\begin{document}

\title{MicronetToNMEA}

\chapter {Introduction}

\section{What is MicronetToNMEA}

MicronetToNMEA is a Teensy/Arduino based project aimed at building a cheap NMEA/Micronet bridge. The initial purpose of this project was to understand Micronet wireless protocol and to be able to record wind and speed data on a PC. The understanding of the protocol went so well that MicronetToNMEA is now doing a lot more than that. It can:

\begin{itemize}
\item Send NMEA stream to your PC/Tablet with data on depth, water speed, wind, magnetic heading, GNSS positioning, speed, time, etc.
\item Send Heading data from the navigation compass to your Micronet's displays (HDG).
\item Send GNSS data to your Micronet displays (LAT, LON, TIME, DATE, COG, SOG).
\item Send navigation data from OpenCPN or qtVlm to your Micronet displays (BTW, DTW, XTE, ETA).
\end{itemize}

\section{What is NOT MicronetToNMEA}

MicronetToNMEA is not waterproof and more generally not reliable. All electronics used in this project are made for hobbyist and are all but robust. In the brutal, wet and salty environment of a boat, it will likely fail quickly. So be careful that MicronetToNMEA should not be used as primary navigation tool. Also note that Micronet wireless protocol has been reverse engineered and that many of its aspects are not yet properly understood. Worse, some understandings we think to be correct might very well be false in some circumstances. If you need state of the art and reliable navigation devices, just go to your nearest Raymarine/TackTick reseller.

\section{Contributors}

\begin{itemize}
\item Ronan Demoment : Main author
\item Dietmar Warning : LSM303 drivers \& Bugfixes
\item Contributors of YBW forum's Micronet thread : \href{https://forums.ybw.com/index.php?threads/raymarines-micronet.539500/}{Micronet Thread}
\end{itemize}

\chapter{Needed hardware and software}

\section{Required hardware}

To work properly, MicronetToNMEA needs at least a Teensy 3.5 board and a CC1101 based breakout board.

\subsection{Teensy 3.5}
Teensy 3.5 has been chosen as the core micro-controller of the MicronetToNMEA system. This choice has been led by one main reason : I had one available when I started investigating Micronet protocol. That's indeed a good reason but with time this board has also proven to be pretty well adapted :

\begin{itemize}
	\item It is small
	\item Teensy software stack is rich and stable
	\item It has a lot of highly configurable peripherals
	\item GPIOs are 5V tolerant (important to connect 5V GNSS !)
	\item It has a MicroSD slot for future recording features
\end{itemize}

In theory, you can port MicronetToNMEA SW to any 32bit Arduino compatible board. Practically, this might be a different story. Several people got into troubles trying to use ESP32 boards. While this is technically feasible, Arduino's library implementation between Teensy \& Esp32 board can be slightly different in some sensitive areas like interrupt handling. This makes porting complex.

Some users successfully used MicronetToNMEA on Teensy 4.0 and 4.1 boards with minimal adaptations.

Teensy boards can be ordered here : \url{https://www.pjrc.com/teensy/}

\subsection{CC1101 board}

CC1101 is absolutely mandatory to MicronetToNMEA. It is the IC which enable RF communication with Micronet/TackTick devices. CC1101 breakout boards are very cheap but the quality of design and components is often less than average. So do not expect to have the same distance performance than an original TackTick device. Be careful when ordering this board since it is designed for a specific range of frequencies (filter and antenna), even if the board is announced to support 434 \& 868 (the IC can, but the antenna filter can not). MicronetToNMEA needs a board designed for 868/915MHz usage. Ordering the wrong board would dramatically reduce operating distance between MicronetToNMEA and TackTick devices. Here is an example of a suitable board: \href{https://www.amazon.fr/laqiya-cc1101-868-MHz-Transmission-Antenne-Transceiver/dp/B075PFQ57G}{868MHz CC1101}

These low-cost boards are often delivered without any documentation, especially pin-out description. In that case CC1101 data-sheet might help : \href{https://www.ti.com/lit/ds/symlink/cc1101.pdf}{CC1101 data-sheet}

\section{Optional hardware}

You can add optional HW to MicronetToNMEA to enhance its capabilities.

\subsection{NMEA0183 GNSS}

If you want to connect a GNSS/GPS to MicronetToNMEA, there is only one important point : it must output localization data on a RS232 link using NMEA0183 format. An example of cheap GNSS which fits the need is the UBLOX NEO-M8N. The NEO-M8N can directly output NMEA stream to its serial output. Avoid too cheap offers from unknown HW sources, this might be counterfeit hardware.

\subsection{LSM303DLH or LSM303DLHC navigation compass breakout board}

Connected to Teensy I2C bus, this IC will allow getting magnetic heading. MicronetToNMEA automatically detect the presence and type of LSM303DLH/DLHC on its I2C bus.

\subsection{HC-06 Bluetooth transceiver}

You can connect HC-06 device to MicronetToNMEA serial NMEA output to easily get a wireless connection to a PC/Tablet. Note that MicronetToNMEA does not configure HC-06 link, it is up to you to configure HC-06 before connecting it.

\section{Required software}

\subsection{Arduino IDE (required)}
Arduino IDE provides gcc-arm compiler and all libraries necessary for MicronetToNMEA. This is the first software you must install.

\subsection{Teensyduino (required)}

Teensyduino is an extension to Arduino IDE which add full support to all Teensy’s board, including Teensy 3.5. It must be installed on top of Arduino IDE to enable compilation for Teensy 3.5.

\section{Optional software}

\subsection{Sloeber (optional)}

If you plan to do more than just compile MicronetToNMEA’s code, you probably need a more serious IDE. Sloeber is an Arduino compatible version of Eclipse. It provides many useful features which will highly improve your productivity. It requires Arduino IDE and Teensyduino to be already installed.

\chapter{Compilation}

\section{With Arduino IDE}

Here are the steps to compile MicronetToNMEA with Arduino IDE:

\begin{itemize}
\item Get the source code from MicronetToNMEA repository (\url{https://github.com/Rodemfr/MicronetToNMEA})
\item Double-click on MicronetToNMEA.ino. This should open Arduino IDE.
\item In Arduino IDE, select Teensy 3.5 target HW with menu “Tools->Board->Teensyduino->Teensy3.5”
\item Go to menu “Tools->Manage Libraries...” and install TeensyTimerTool library
\item Click on “Verify” button in the button bar, this should compile the project without error.
\item Connect your Teensy 3.5 board onto USB port of your PC and Click “Upload” button to upload MicronetToNMEA binary into Teensy flash memory
\end{itemize}

\section{With Sloeber}

Here are the steps to compile MicronetToNMEA with Sloeber IDE:

\begin{itemize}
\item Before trying to compile with Sloeber, you must have successfully compiled with Arduino IDE
\item Start Sloeber and create your Workspace as requested Select menu “File->New->Arduino Sketch"
\item In Sloeber, select menu Arduino->Preferences
\item Add Arduino's library and hardware path in the path lists
\item Exit the panel by clicking "Apply and Close"
\item Select menu File->New-Arduino Sketch
\item Name your project "MicronetToNMEA"
\item Don't use default project location and set the location to your git cloned repository of MicronetToNMEA
\item Click "Next"
\item Select Teensy's platform folder
in the corresponding drop down menu
\item Select "Teensy 3.5" board
\item Select "Faster" optimization
\item Select "Serial" USB Type
\item Select 120MHz CPU Speed
\item Click "Next"
\item Select "No file" as code
\end{itemize}

Your project should be compiling now.

Note that Sloeber can be somewhat picky with tool-chain or library paths. So don’t be surprised if you have to handle additional issues to compile with it. The effort is worth, code productivity with Eclipse is way beyond Arduino IDE.

\section{Compile time configuration}

By default, MicronetToNMEA is configured for a specific HW layout. This means that it is configured to be connected through specific SPI, I2C or GPIO pins to various boards. This configuration can be changed to some extent to adapt your own needs. The file bearing this configuration is “BoardConfig.h”.Note that no coherency check are made in the software. It is your responsibility to provide a reachable configuration (i.e. not to connect SPI wires to non SPI capable pins). Table {\ref{table:configswitches}} lists all available switches and their meaning.

\begin{table}[h]
	\begin{tabular}{|l|p{12cm}|}
		\hline
		FREQUENCY\_SYSTEM & Defines which frequency range is used by your Micronet network (0=868MHz, 1=915MHz)\\
		\hline
		NAVCOMPASS\_I2C & Sets the I2C bus to which the navigation compass (i.e. LSM303DLH(C)) is connected. Defined as per “Wiring” library definition (Wire0, Wire1, etc.)\\
		\hline
		CS0\_PIN & Defines SPI Chip Select line connected to RF IC\\
		\hline
		MOSI\_PIN & Defines MOSI pin of SPI bus connected to RF IC\\
		\hline
		MISO\_PIN & Defines MISO pin of SPI bus connected to RF IC\\
		\hline
		SCK\_PIN & Defines SCK pin of SPI bus connected to RF IC\\
		\hline
		GDO0\_PIN & Defines GDO0 pin of SPI bus connected to RF IC\\
		\hline
		LED\_PIN & Defines the pin driving the LED, which is used for error signaling\\
		\hline
		GNSS\_UBLOXM8N & Enable automatic configuration of UBLOX M8N GPS (0=disabled, 1=enabled)\\
		\hline
		GNSS\_SERIAL & Defines on which serial port is connected the NMEA GNSS (Serial, Serial1, Serial2, etc.)\\
		\hline
		GNSS\_BAUDRATE & Defines GNSS bit-rate in baud\\
		\hline
		GNSS\_CALLBACK & Defines the name of the callback function called when new bytes arrive on the
		configured serial port\\
		\hline
		GNSS\_RX\_PIN & Defines serial RX pin connected to NMEA GNSS TX pin\\
		\hline
		GNSS\_TX\_PIN & Defines serial TX pin connected to NMEA GNSS RX pin\\
		\hline
		USB\_NMEA & Defines which serial port is connected to USB serial converter\\
		\hline
		USB\_BAUDRATE & Defines baud rate of USB serial converter\\
		\hline
		WIRED\_NMEA & Defines which serial port is connected to the wired NMEA connection\\
		\hline
		WIRED\_BAUDRATE & Defines baud rate of the wired NMEA connection\\
		\hline
		WIRED\_RX\_PIN & Defines serial RX pin used for wired NMEA\\
		\hline
		WIRED\_TX\_PIN & Defines serial TX pin used for wired NMEA\\
		\hline
		CONSOLE & Defines on which serial port is displayed the console (can be USB\_CONSOLE or WIRED\_SERIAL). Can be on the same serial link than NMEA\_IN and NMEA\_OUT\\
		\hline
		NMEA\_OUT & Defines on which serial port to output NMEA stream. (can be USB\_CONSOLE or WIRED\_SERIAL). Can be on the same serial link than CONSOLE and NMEA\_IN\\
		\hline
		NMEA\_IN & Defines on which serial port to read input NMEA stream. (can be USB\_CONSOLE or WIRED\_SERIAL). Can be on the same serial link than CONSOLE and NMEA\_OUT\\
		\hline
\end{tabular}
	\caption{Configuration switches in BoardConfig.h}
\label{table:configswitches}
\end{table}

\chapter{Installation}

Teensy board must be connected to other boards with the same scheme than you have defined in “BoardConfig.h”. No check is made by MicronetToNMEA software to verify that your configuration is matching your actual connections. You must carefully verify that you properly connected the various devices since wrong connections can possibly damage your hardware, especially with respect to power supply connections which are mixing 3.3 \& 5V levels.

\section{Power supply}

The first and most important connection to build is the power supply. You have two options there, you can either :

\begin{itemize}
	\item Power the system via USB
	\item Power the system using external DC power source
\end{itemize}

\subsection{Power via USB}

This is the most straightforward way to power the system : just plug an USB cable in the Teensy connector and it will be powered by the connected PC. Teensy board is equipped with a voltage regulator which provides 3.3V. This 3.3V voltage can be used to power other boards of the system.
Be careful that USB 2.0 limits 5V output current to 500mA, but you should be even more careful since Teensy's regulator recommends not to exceed 250mA for 3.3V. So you must take care that your system does not exceed these limits.
As an example, table \ref{table:boardconsumption} shows maximum current values for various boards.

\begin{table}[h]
	\begin{tabular}{|l|c|c|l|}
		\hline
		Board & Voltage source & Max current & Comment \\
		\hline
		Teensy 3.5 & 3.3V & 50mA & CPU ruuning at 120MHz\\
		CC1101 & 3.3V & 40mA & RF at 868MHz\\
		NEO M8N GNSS & 5V & 45mA & M8N is 3.3V but the board is 5V\\
		LSM303DLH(C) & 3.3V & 10mA & Unspecified in datasheet, value assumed\\
		HC06 Bluetooth transceiver & 3.3V & 40mA & Peak during pairing\\
		\hline
	\end{tabular}
	\caption{Current consumption of typical boards}
	\label{table:boardconsumption}
\end{table}
USB powering is especially useful when you plan to output NMEA through USB-serial. In this case, the connected PC/Tablet will provide power to the system and when MicronetToNMEA is not needed anymore, just unplug the USB cable to power-off the system.

\subsection{Power with an external DC source}
While USB powering is easy to setup, it not a common source of power in a boat. It is more usual to get two wires with an unstable battery voltage between 11V and 15V. In that case, you will need a voltage regulator or a DC-DC converter which will be used to produce a stable 5V for the system. This 5V source can then be connected to the Vin pin of Teensy 3.5. The on-board regulator will then produce 3.3V from this input.
\linebreak
When Vin pin is connected to an external source of power, you must not connect an USB cable to avoid short circuit between Vin and Vusb which are connected together on Teensy by default. The Vin pin can handle voltages from 3.6 to 6V but it is strongly recommended to use 5V here. This way, if you accidentally connect a USB cable while powering Vin, there will be no short circuit.

\begin{figure}
\centering
\includegraphics{MicronetToNMEA_DC_Power.png}
	\caption{Powering Teensy with a DC-DC converter}
\label{figure:dcpower}
\end{figure}

\section{Connecting CC1101}

CC1101 uses 3.3V voltage so you can connect Teensy's 3.3V \& GND pins to CC1101's VCC \& GND. MOSI(SI), MISO(SO), CS0 and GD0 must be connected as per your BoardConfig.h definitions. Note that GD2 isn't used by MicronetToNMEA and doesn't need to be connected to Teensy. Figure \ref{figure:cc1101} shows how to connect CC1101 with the default configuration.

\begin{figure}[h]
	\centering
	\includegraphics{MicronetToNMEA_CC1101.png}
	\caption{Connecting Teensy and CC1101}
	\label{figure:cc1101}
\end{figure}

\section{Connecting LSM303}

LSM303DLH(C) uses 3.3V voltage so you can connect Teensy's 3.3V \& GND pins to CC1101's VCC \& GND. In addition SDA \& SCL must be connected as per your BoardConfig.h definitions. Note that DRDY, I1 \& I2 don't need to be connected. Figure \ref{figure:lsm303} shows how to connect LSM303DLH(C) with the default configuration.

\begin{figure}[h]
	\centering
	\includegraphics{MicronetToNMEA_LSM303.png}
	\caption{Connecting Teensy to LSM303}
	\label{figure:lsm303}
\end{figure}

\section{Connecting GNSS}

Unlike CC1101 or LSM303DLH(C), GNSS/GPS boards are often requiring 5V VCC as power voltage. So you have to connect it directly to DC-DC Converter's output. You should check however that your GNSS board is not 3.3V powered, in which case you should use one of Teensy 3.3V pin.
TX and RX pins must then be connected respectively on RX and TX of Teensy's UART. Note that Teensy 3.5 is 5V tolerant, so you can connect GNSS directly even if it is using 5V output. Figure \ref{figure:gnss} shows how to connect GNSS for the default configuration.

\begin{figure}[h]
	\centering
	\includegraphics{MicronetToNMEA_GNSS.png}
	\caption{Connecting Teensy and GNSS}
	\label{figure:gnss}
\end{figure}

MicronetToNMEA can connect to a wide variety of GNSS. You just have to configure the GNSS with the correct parameters before connecting it. GNSS has to output an NMEA compatible stream at the same bit-rate than specified in BoardConfig.h. MicronetToNMEA can automatically configure the GNSS if it is a UBLOX Neo M8N. Just enable GNSS\_UBLOXM8N option in BoardConfig.h.

GNSS should output the following sentences :
\begin{itemize}
	\item GGA : Position
	\item RMC : Time
	\item VTG : Track and speed
\end{itemize}

\section{Connecting HC-06 modules}

When MicronetToNMEA is configured to send its console and/or NMEA output to a standard wired UART (i.e not USB), you can consider connecting a HC-06 Bluetooth transceiver to easily get wireless connectivity to your PC/Tablet. HC-06 is powered with 5V but can handle 3.3V signals. Only VCC, GND, RXD \& TXD need to be connected. Figure \ref{figure:hc06} shows how to connect HC-06 with the default configuration.

\begin{figure}[h]
	\centering
	\includegraphics{MicronetToNMEA_HC06.png}
	\caption{Connecting Teensy and HC-06}
	\label{figure:hc06}
\end{figure}

As for the GNSS, MicronetToNMEA does not configure HC-06 itself. It is your responsibility to configure HC-06 properly (i.e. with parameters matching BoardConfig.h) prior to connecting it to Teensy.

\section{Recommendations}

\subsection{RF performance}
Micronet devices cannot legally exceed a handful of milliwatts of transmit power. As a consequence the operational range will hardly exceed 15-20m. It is important not to put metal objects or panels between your various devices. This would dramatically reduce of the operational range of MicronetToNMEA.
You should also be careful if you are racing and using carbon sails. If the sails are between your wind vane and MicronetToNMEA, you will likely not receive wind data. Carbon disturbs RF transmissions.
It is generally considered not to be a good idea to use wireless electronics in a carbon boat or with carbon sails. Fiberglass doesn't attenuates the signal (only a little), so it is safe to attach your device inside a GRP hull.

\subsection{Magnetic compass disturbances}
\label{compass-recommendations}
If your are using LSM303 to get magnetic heading, it is very important to put your MicronetToNMEA assembly far from other electrical devices, especially those carrying a bit of power. This can highly disturbs magnetic compass. As an example, a running 24" TV monitor can deviate the compass by 20° at 50cm and still a few degrees at 1m.
Also, metal must be avoided : it deviates magnetic field. The bigger the piece of metal is the farthest you should put your device. In a boat you should avoid the keel, batteries and the inboard engine. At a smaller level, you should keep LSM303 board away from DC-DC converter if you can.

\subsection{Magnetic compass calibration}
An electronic compass must be properly calibrated to produce good values. The theory tells us that a perfect calibration would need to have your device attached in its final position and to spin your boat around the 3 axis. This is generally not something we want to do. To achieve a good calibration with less complex operations, the best is to calibrate MicronetToNMEA outside your boat, far from any metal or electronic device. You should also keep it as far as possible of its power supply. Once calibrated, get it back to your boat and fix it in a proper place, following the above recommendations. This should give good results.

\chapter{Usage}

\section{Configuring MicronetToNMEA}

\subsection{Connecting to the console}

To be able to configure MicronetToNMEA, you must have access to the configuration menu displayed on the serial console. By default, this console is connected to the USB serial port of Teensy. So you just have to find a terminal software to connect to it. A good open source terminal is \href{http://www.teraterm.org/}{Tera Term}. Note that when using Teensy USB connection, you can use any baud-rate, this as no real effect.

At the first power-up, before having configured MicronetToNMEA, console should go directly to the configuration menu. If this is not the case, it means that a configuration as already been written in EEPROM and that MicronetToNMEA has automatically switched to "NMEA conversion mode". In this case, you just have to press \emph{<ESC>} key to come back to the configuration menu.

The menu should look like this :

\begin{verbatim}
*** MicronetToNMEA ***

0 - Print this menu
1 - General info on MicronetToNMEA
2 - Scan Micronet networks
3 - Attach converter to a network
4 - Start NMEA conversion
5 - Scan surrounding Micronet traffic
6 - Calibrate RF frequency
7 - Calibrate magnetometer
\end{verbatim}

\subsection{Attaching your Micronet network to MicronetToNMEA}

We need to identify your Micronet network for MicronetToNMEA to be able to recognize it and to discard other devices that may be in you vicinity. You will typically find several networks when in a marina with many sailing boats.
To identify your network, you have to power-up your Micronet display and to ensure that your master device, the one you used to power-up the network, is close to MicronetToNMEA. Once done, enter \emph{"2 - Scan Micronet networks"} menu by pressing 2 in the console. This will start a five second scanning sequence which listens to every network in the area. At the end of this sequence, every network found will be listed in the console. If there are several networks, they are listed in decreasing order of power.

\begin{verbatim}
Network - 83038F54 (very strong)
Network - 810278A6 (low)
\end{verbatim}

Yours is very likely at the top of the list with mention "very strong". This is the signal strength. For each network you get a hexadecimal number which is the so called "Network ID". Write it on a piece of paper (\emph{83038F54} in our example).
Now enter menu \emph{"3 - Attach converter to a network"} by pressing 3, and type your network ID when requested. This will attach MicronetToNMEA to your Micronet devices.
Once done, MicronetToNMEA will save this value to EEPROM and will remember it. You don't need to do this operation at each start-up. Now that MicronetToNMEA is attached, it will automatically go to NMEA conversion mode at the next power-up.

\subsection{Calibrating CC1101 RF frequency}

CC1101 board uses a crystal to generate its frequencies. Crystal frequency is more or less precise and needs to be calibrated for CC1101 to generate exactly the expected frequency. Each crystal needs to be calibrated independently. This calibration is very important because it directly influence the range performance of the RF system (i.e. the maximum distance at which MicronetToNMEA can detect other devices). When you buy a Tacktick device, this calibration has already been done at factory. But when you build a MicronetToNMEA system, you need to do it yourself.

Menu \emph{"6 - Calibrate RF frequency"} does this calibration. Enter it by pressing key 6. A text will explain the procedure : power-up your Micronet network and place the master device (the one you used to power-up the network) very close to MicronetToNMEA (less than one meter). Now press any key and let the calibration proceed. During 2 minutes, you will see dots and stars appear on the console looking like this :

\begin{verbatim}
...................*.**********************.**.....................................
\end{verbatim}

Each character represents a tested frequency where MicronetToNMEA verifies if it receives data from the Master device. A dot means that nothing was received, a star means that a message was received. The center of the starred area is the best reception frequency and MicronetToNMEA will use this one to get the best RF performance. Your crystal is calibrated.
At the end of the procedure, you will be asked if you want to save the new calibration value to EEPROM. Just answer yes and your calibration will be done once for all.

\subsection{Calibrating navigation compass}

Navigation compass needs to be calibrated on its 3 axis to produce correct heading measurements. An uncalibrated compass will give totally wrong values and not just by a few degrees.

The base principle of calibration consists in determining the minimum and maximum values of earth's magnetic field onto each of the three axis of the sensor. This way, heading calulation will be able to remove any bias on the 3 axis. this biais is produced by The LSM303 itlself, but also by surrounding electronics (Teensy, GNSS, etc.).

Before starting calibration, you must first prepare your environment as explained in \ref{compass-recommendations}. Once ready, you can enter menu \emph{"7 - Calibrate magnetometer"}.

This menu will produce a permanent output of calibration values looking like this :

\begin{verbatim}
(0.2 -0.35 -0.17)
[0.20 0.98] [0.12 1.0] [-0.25 0.99]
\end{verbatim}

The first line gives the current magnetic field on each of the three axis : X, Y and Z.
The second line gives the two calibration parameters for each axis : center value ((max + min) / 2) and axis amplitude (max - min).

Your objective for a successful calibration is to maximize/minimize readings on each axis. To do this, you must first choose one axis and then rotate your MicronetToNMEA device until this axis is aligned with earth's magnetic field vector. You should see the corresponding reading on the first line maximum (or minimum), while the two others are zero or close to zero.
You must repeat this operation a second time for the same axis but for opposite values (i.e. you must minimize the reading). Each time you reach the maximized/minimized value, you don't need to do anything, MicronetToNMEA will memorize the min/max values for each axis.

Note that the magnetic field vector is pointing more or less to the north, but not horizontally, it is pointing somewhat down to earth in north hemisphere and up to the sky in south hemisphere.

You must reproduce this operation for all 3 axis to complete calibration.
A successful calibration should lead to very close amplitude values for all 3 axis, on the second line. Center values can however be very different.

Once achieved, just press \emph{<ESC>} and tell MicronetToNMEA to save the value in EEPROM. Your compass is calibrated.

\subsection{Starting NMEA conversion}

Menu \emph{"4 - Start NMEA conversion"} actually start Micronet/NMEA conversion as suggested by the name. Once you enter in this mode MicronetToNMEA will output NMEA sentences to the NMEA\_OUT link. By default, NMEA\_OUT is routed to the same serial link than the console (USB serial). It means that you will immediately see NMEA sentences beeing written onto console. More silently, MicronetToNMEA also decodes any incoming NMEA sentence from NMEA\_IN serial (the same than NMEA\_OUT and CONSOLE by default) and transmits it to your Micronet network.
You need to be in this mode to see decoded GNSS or NMEA data displayed onto your Micronet displays. You also need to be in this mode for the configuration parameters modified onto your micronet displays to be memorized by MicronetToNMEA (e.g. speed factor, sounder offset, etc.).
If attached to a network, MicronetToNMEA will automatically switch to this mode when powered-up. You can leave by just pressing \emph{<ESC>} in the console.
\\
\\
Here is a summary of all actions realized by MicronetToNMEA in this mode :
\begin{itemize}
	\item Collect and decode NMEA sentences from GNSS\_NMEA link
	\item Collect and decode NMEA sentences from NMEA\_IN link
	\item Forward NMEA sentences from GNSS\_NMEA link to NMEA\_OUT link
	\item Collect and decode data from Micronet devices
	\item Send all data collected from NMEA links to Micronet network
	\item Send all data collected from Micronet devices to NMEA\_OUT link
	\item Calculate heading from LSM303DLH(C), if any
	\item Send heading data to Micronet network and NMEA\_OUT link
\end{itemize}

\subsection{Supported NMEA sentences}

The following table summarizes all supported NMEA sentences and how they are decoded or encoded :
\begin{table}[h]
	\begin{tabular}{|l|c|c|c|c|}
		\hline
		\textbf{Sentence} & \textbf{NMEA\_IN} & \textbf{NMEA\_OUT} & \textbf{From Micronet} & \textbf{To Micronet} \\
		\hline
		RMB & Decoded & & & XTE DTW BTW VMGWP \\
		\hline
		RMC & Decoded & Forwarded from NMEA\_IN & & TIME DATE \\
		\hline
		GGA & Decoded & Forwarded from GNSS\_SERIAL & & LAT LON \\
		\hline
		VTG & Decoded & Forwarded from GNSS\_SERIAL & & COG SOG \\
		\hline
		MWV (R) & & Encoded from Micronet & AWA AWS & \\
		\hline
		MWV (T) & & Encoded from Micronet & TWA TWS & \\
		\hline
		DPT & & Encoded from Micronet & DPT & \\
		\hline
		MTW & & Encoded from Micronet & STP & \\
		\hline
		VLW & & Encoded from Micronet & LOG TRIP & \\
		\hline
		VHW & & Encoded from Micronet & SPD & \\
		\hline
		HDG & & Encoded from LSM303 & & HDG \\
		\hline
	\end{tabular}
	\caption{Supported NMEA sentences}
	\label{table:nmeasentences}
\end{table}

\end{document}